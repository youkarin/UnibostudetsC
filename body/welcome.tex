
% \thispagestyle{empty}                   %elimina il numero della pagina
% \topmargin=6.5cm                        %imposta il margina superiore a 6.5cm
% \raggedleft                             %incolonna la scrittura a destra
% \large                                  %aumenta la grandezza del carattere
%                                         %   a 14pt
% \em                                     %emfatizza (corsivo) il carattere
% Dedico questa tesi alla mia famiglia e \\ tutti coloro che hanno creduto in me.                    %\ldots lascia tre puntini
% \newpage                                %va in una pagina nuova
% %
% %%%%%%%%%%%%%%%%%%%%%%%%%%%%%%%%%%%%%%%%
% 


\chapter{博洛尼亚欢迎你}                 %crea l'introduzione (un capitolo

\section{学联致辞}

各位新同学:欢迎来到博洛尼亚!欢迎成为博洛尼亚中国留学生群体的一员!博洛尼亚大学是意大利国际化程度最高的大学之一,也是最早出现中国留学生身影的地方。随着中意两国政府教育合作的不断深化,从2004年开始博洛尼亚大学不断增加在中国的招生规模,至今已有数千名学生走进博洛尼亚大学。2021/22学年,注册在读的本科和硕士中国学生共914名、博士共67名。博洛尼亚大学中国学联(以下简称学联),即Associazione di Studenti e Studiosi Cinesi dell'Università di Bologna(简称ASSCUBO),是第一个由中国学生在意大利注册的地方学联。2009年10月,学联举办了第一次全意大利规模的中国留学生活动:首届全意中国留学生篮球联赛;在汶川、拉奎拉和玉树地震灾害后,学联分别举行了三次赈灾募捐活动,总共募捐 8819.38 欧元;2010年5月,学联代表中国学生第一次参加博洛尼亚大学学生会的选举,产生了意大利大学史上第一名中国留学生议员;学联也多次接受意大利媒体的访问,为增强意大利社会对中国的了解与认识起到了积极的作用。学联以服务留学生为己任,为了更好地做到这一点,学联建立了完善的公众信息平台,通过微信公众号、微博和网站等多个渠道提供信息和服务;和商户合作为学生提供更多便利;作为一个留学生展现自我和锻炼才能的舞台,所有加入学联的干事都是义务为大家服务的,促进了博大中国留学生友爱互助的和谐氛围的形成,并成为优良传统继承至今;为了增强海外的华人的凝聚力,服务更多莘莘学子,学联开展形式多样的文体活动,如意语角、篮球赛以及各式娱乐活动等;每逢中国传统佳节学联也会举办活动,邀请一些当地华人和意大利朋友参加,借此增强与中意的沟通与合作,积极弘扬中国文化,并帮助同学们更好地融入当地社会。目前,学联接受中华人民共和国驻意大利大使馆科教处的指导,并已在意大利内政部注册成为官方组织。学联由主席、副主席、秘书处、学习部、文体部、外联部和宣传部组成,学联的各个部门每年都招收干事与成员,欢迎大家的踊跃参与申请。

\section{亚洲学院(Asia Institute)}

亚洲学院的宗旨在于推进有关亚洲方面的研究、培训、文化以及协助企业成长。
亚洲学院于2021年3月创立,有5个始创成员:博洛尼亚大学、艾米利亚-罗马涅大区、博洛尼亚市政府、博洛尼亚展览会和Confindustria Emilia Area Centro企业联合会。
亚洲学院协会的前身为创立于2005年10月的中国学院协会(Collegio di Cina),其拥有9个创始会员:除了博洛尼亚大学之外,还有艾米利亚-罗马涅大区政府、博洛尼亚省政府、博洛尼亚市政府、博洛尼亚商会、地区商会联盟、博洛尼亚展览会、博洛尼亚工业协会、博洛尼亚中小企业协会、博洛尼亚手工业全国联合会、Alma Mater基金会。它们代表了艾米利亚-罗马涅大区最重要的行政、文化、经济、工业及社会机构。在成立的14年间,中国学院协会在中意双方文化交流方面起到了重要作用。\\
亚洲学院协会将是一个接待和培训来博洛尼亚大学求学的亚洲学生的中心,也将会是所有研究亚洲的学者们交流的中心。亚洲学院协会也将会是艾米利亚-罗马涅大区与亚洲关系的推动者,会举行与当地市民和企业有关的活动。
\\
\noindent 
开放时间:\\
周一至周五  9:30-13:30\\
地点:Palazzo Paleotti (3楼),Via Zamboni, 25, 40126, Bologna\\
官方网站:https://site.unibo.it/asiainstitute/it\\
Facebook:https://www.facebook.com/asiainstituteassociazione/\\
新浪微博:www.weibo.com/collegiodicina\\
秘书处邮箱:asiainstitute@unibo.it\\
电话: 051 2099767 \\
亚洲学院中文导师  吴殷哲\\
邮箱:tutor.asiainstitute@unibo.it\\
\section{博洛尼亚大学孔子学院}

意大利博洛尼亚大学孔子学院,由中国人民大学与博洛尼亚大学于2009年3月2日共建成立。现任外方院长为博洛尼亚大学法学院Marina Timoteo教授,中方院长为中国人民大学哲学院许涤非教授。博洛尼亚大学孔子学院下设一所孔子课堂和若干个汉语教学点,设有汉语初级课程、中级课程、高级课程和中国文化体验课程。博洛尼亚大学孔子学院近年来举办的一系列精品中国文化推广活动在当地社会引起了强烈反响。  \\
\noindent 
\\
秘书处工作时间:周一、周二和周四 9:00-13:00,周三和周五 14:00-18:00\\
地址:Palazzo Paleotti (2楼) ,Via Zamboni 25, 40126 Bologna\\
邮箱:istituitoconfucio@unibo.it\\
电话:051 2088537\\
官方网站:http://www.istitutoconfucio.unibo.it\\
要访问孔子学院秘书处,请先通过邮箱预约。\\
\\
中方院长:许涤非教授\\
邮箱:difei.xu@unibo.it\\
意方院长:Marina Timoteo教授\\
邮箱:marina.timoteo@unibo.it\\


